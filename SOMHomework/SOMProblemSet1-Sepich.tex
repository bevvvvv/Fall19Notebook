\documentclass[]{article}
\usepackage{lmodern}
\usepackage{amssymb,amsmath}
\usepackage{ifxetex,ifluatex}
\usepackage{fixltx2e} % provides \textsubscript
\ifnum 0\ifxetex 1\fi\ifluatex 1\fi=0 % if pdftex
  \usepackage[T1]{fontenc}
  \usepackage[utf8]{inputenc}
\else % if luatex or xelatex
  \ifxetex
    \usepackage{mathspec}
  \else
    \usepackage{fontspec}
  \fi
  \defaultfontfeatures{Ligatures=TeX,Scale=MatchLowercase}
\fi
% use upquote if available, for straight quotes in verbatim environments
\IfFileExists{upquote.sty}{\usepackage{upquote}}{}
% use microtype if available
\IfFileExists{microtype.sty}{%
\usepackage{microtype}
\UseMicrotypeSet[protrusion]{basicmath} % disable protrusion for tt fonts
}{}
\usepackage[margin=1in]{geometry}
\usepackage{hyperref}
\hypersetup{unicode=true,
            pdftitle={Science of Music: Problem Set 1},
            pdfborder={0 0 0},
            breaklinks=true}
\urlstyle{same}  % don't use monospace font for urls
\usepackage{graphicx,grffile}
\makeatletter
\def\maxwidth{\ifdim\Gin@nat@width>\linewidth\linewidth\else\Gin@nat@width\fi}
\def\maxheight{\ifdim\Gin@nat@height>\textheight\textheight\else\Gin@nat@height\fi}
\makeatother
% Scale images if necessary, so that they will not overflow the page
% margins by default, and it is still possible to overwrite the defaults
% using explicit options in \includegraphics[width, height, ...]{}
\setkeys{Gin}{width=\maxwidth,height=\maxheight,keepaspectratio}
\IfFileExists{parskip.sty}{%
\usepackage{parskip}
}{% else
\setlength{\parindent}{0pt}
\setlength{\parskip}{6pt plus 2pt minus 1pt}
}
\setlength{\emergencystretch}{3em}  % prevent overfull lines
\providecommand{\tightlist}{%
  \setlength{\itemsep}{0pt}\setlength{\parskip}{0pt}}
\setcounter{secnumdepth}{5}
% Redefines (sub)paragraphs to behave more like sections
\ifx\paragraph\undefined\else
\let\oldparagraph\paragraph
\renewcommand{\paragraph}[1]{\oldparagraph{#1}\mbox{}}
\fi
\ifx\subparagraph\undefined\else
\let\oldsubparagraph\subparagraph
\renewcommand{\subparagraph}[1]{\oldsubparagraph{#1}\mbox{}}
\fi

%%% Use protect on footnotes to avoid problems with footnotes in titles
\let\rmarkdownfootnote\footnote%
\def\footnote{\protect\rmarkdownfootnote}

%%% Change title format to be more compact
\usepackage{titling}

% Create subtitle command for use in maketitle
\newcommand{\subtitle}[1]{
  \posttitle{
    \begin{center}\large#1\end{center}
    }
}

\setlength{\droptitle}{-2em}

  \title{Science of Music: Problem Set 1}
    \pretitle{\vspace{\droptitle}\centering\huge}
  \posttitle{\par}
  \subtitle{Due Monday Sept 2; Joseph Sepich (jps6444)}
  \author{}
    \preauthor{}\postauthor{}
    \date{}
    \predate{}\postdate{}
  

\begin{document}
\maketitle

Directions: Save this worksheet with the filename
``SOMProblemSet1-(lastname)''. Enter your answers directly into the
document and save, then upload it into Canvas. Clearly show your work.
If you are using an equation, write the equation down, then plug in
values you are using on the next line. When you are finished with a
problem, draw a box around your final result or highlight it. Leave at
least a couple of spaces on the paper before starting your next problem.
Every answer should include the appropriate units. Remember to submit
your homework by 10pm Monday. No homework will be accepted late, so get
it done early.

A note about collaboration vs.~plagiarism As with all homework
assignments, you should work on each problem on your own. After having
done that, you may seek guidance from other students or professors. But
that help should come in the form of a discussion of the problem, not
simply looking at the other person's work. You should then go and work
the problem by yourself. You may compare final answers. Whenever you
received significant assistance, it is your obligation to write a note
on your solution to that individual problem that gives credit to that
other individual.\\
Final note on collaboration: You may not get help from any written work
of students that have previously taken this class. Of course this is on
your honor\ldots{}and it is our expectation!

\section{Convert the following speeds into kilometers per hour (km/hr)
from miles per hour
(mph):}\label{convert-the-following-speeds-into-kilometers-per-hour-kmhr-from-miles-per-hour-mph}

We know that the conversion rate from mile to kilometer is 1 mile to
1.60934 kilometers.

\subsection{a) 35 mph}\label{a-35-mph}

\[\frac{35 mi}{1 hour} * \frac{1.60934 km}{1 mi} = 56.3269 mph\]

This means that 35 miles per hour is equivalent to 56.3269 kilometers
per hour.

\subsection{b) 75 mph}\label{b-75-mph}

We can once again multiply 75 by 1.60934 to get 75 mph to be equivalent
to 120.701 kilometers per hour.

\subsection{c) 760 mph}\label{c-760-mph}

We can once again multiply 760 by 1.60934 to get 760 mph to be
equivalent to 1223.1 kilometers per hour.

\section{Convert the following speeds into meters per second
(m/s):}\label{convert-the-following-speeds-into-meters-per-second-ms}

We know that the conversion rate from mile to meters is 1 mile to
1609.34 meters. We also know that one hour is equivalent to 360 seconds.
Let's use the same conversion method from before.

\subsection{a) 35 mph}\label{a-35-mph-1}

\[\frac{35 mi}{1 hour} * \frac{1609.34 m}{1 mi} * \frac{1 hour}{3600 sec} = \frac{35 mi}{1 hour} * 0.447 = 15.6464 m/s\]

This means that 35 miles per hour is equivalent to 15.6464 meters per
second.

\subsection{b) 75 mph}\label{b-75-mph-1}

We can once again multiply 75 by 0.447 to get 75 mph to be equivalent to
33.528 meters per second.

\subsection{c) 760 mph}\label{c-760-mph-1}

We can once again multiply 760 by 0.447 to get 760 mph to be equivalent
to 339.75 meters per second.

\section{Problem 3}\label{problem-3}

Pretend you are 6 kilometers (km) from where a lightning bolt strikes.
The cracking sound produced will travel at the speed of sound, whereas
the light will travel at the speed of light (300,000,000 m/s).

For this problem I will use the value of the speed of sound to be 343.6
m/s.

\subsection{a) How long does it take the sound to get to you? (Assume
that the speed of sound is the value given on page 9 of our text by Lapp
for room
temperature.)}\label{a-how-long-does-it-take-the-sound-to-get-to-you-assume-that-the-speed-of-sound-is-the-value-given-on-page-9-of-our-text-by-lapp-for-room-temperature.}

Using our value for the speed of sound we know that the sound will
travel 6 km at 343.6 m/s. This means that it will be travelling at
0.3436 km/s. Using our formula \(d = rt\) we can calulate t (or time) by
\(t = \frac{d}{r}\). This will give us \(\frac6{0.3436} = 17.4622\). The
sound will take 17.4622 seconds to reach you.

\subsection{b) How long does it take the light to get to
you?}\label{b-how-long-does-it-take-the-light-to-get-to-you}

Using the value of 300,000,000 m/s that is given we can change this to
be 300,000 km/s. The light will travel at this speed for 6 km, so we can
the equation again \(\frac6{300,000} = 0.00002\). This means the light
will get to you in 0.00002 seconds. This is almost instantly versus the
17 seconds the sound takes.

\section{Problem 4}\label{problem-4}

A conductor taps on his music stand and hears an echo from the back of
the auditorium 0.4 seconds later. Given the speed of sound, what is the
length of the auditorium? (Assume that the room is dry and has a
temperature of 20C -- approximately ``Room temperature.'')

\section{Miscellaneous Questions}\label{miscellaneous-questions}

\subsection{a) How many square inches are in a square
foot?}\label{a-how-many-square-inches-are-in-a-square-foot}

\subsection{b) How many cubic centimeters (cm3) in a cubic meter
(m3)?}\label{b-how-many-cubic-centimeters-cm3-in-a-cubic-meter-m3}

\subsection{c) How many milliliters (ml) in a liter
(l)?}\label{c-how-many-milliliters-ml-in-a-liter-l}

\subsection{d) How many cm3 in a ml?}\label{d-how-many-cm3-in-a-ml}

\subsection{e) How many liters in a cubic
meter?}\label{e-how-many-liters-in-a-cubic-meter}

\section{Problem 6}\label{problem-6}

The Blue Band drum major is standing with you in the end zone. He
signals two drummers to start tapping the beat at the speed that he is
indicating with his hands -- one drumstroke per second. One drummer is
right in front of you and the other one is 100 yards away from in the
other end zone. So, given the relatively slow speed of sound, their
tapping will sound out of phase to you. By how many milliseconds will
the two placements of beats differ?


\end{document}
