\documentclass[]{article}
\usepackage{lmodern}
\usepackage{amssymb,amsmath}
\usepackage{ifxetex,ifluatex}
\usepackage{fixltx2e} % provides \textsubscript
\ifnum 0\ifxetex 1\fi\ifluatex 1\fi=0 % if pdftex
  \usepackage[T1]{fontenc}
  \usepackage[utf8]{inputenc}
\else % if luatex or xelatex
  \ifxetex
    \usepackage{mathspec}
  \else
    \usepackage{fontspec}
  \fi
  \defaultfontfeatures{Ligatures=TeX,Scale=MatchLowercase}
\fi
% use upquote if available, for straight quotes in verbatim environments
\IfFileExists{upquote.sty}{\usepackage{upquote}}{}
% use microtype if available
\IfFileExists{microtype.sty}{%
\usepackage{microtype}
\UseMicrotypeSet[protrusion]{basicmath} % disable protrusion for tt fonts
}{}
\usepackage[margin=1in]{geometry}
\usepackage{hyperref}
\hypersetup{unicode=true,
            pdftitle={Data Structures and Algorithms Homework 1},
            pdfborder={0 0 0},
            breaklinks=true}
\urlstyle{same}  % don't use monospace font for urls
\usepackage{graphicx,grffile}
\makeatletter
\def\maxwidth{\ifdim\Gin@nat@width>\linewidth\linewidth\else\Gin@nat@width\fi}
\def\maxheight{\ifdim\Gin@nat@height>\textheight\textheight\else\Gin@nat@height\fi}
\makeatother
% Scale images if necessary, so that they will not overflow the page
% margins by default, and it is still possible to overwrite the defaults
% using explicit options in \includegraphics[width, height, ...]{}
\setkeys{Gin}{width=\maxwidth,height=\maxheight,keepaspectratio}
\IfFileExists{parskip.sty}{%
\usepackage{parskip}
}{% else
\setlength{\parindent}{0pt}
\setlength{\parskip}{6pt plus 2pt minus 1pt}
}
\setlength{\emergencystretch}{3em}  % prevent overfull lines
\providecommand{\tightlist}{%
  \setlength{\itemsep}{0pt}\setlength{\parskip}{0pt}}
\setcounter{secnumdepth}{5}
% Redefines (sub)paragraphs to behave more like sections
\ifx\paragraph\undefined\else
\let\oldparagraph\paragraph
\renewcommand{\paragraph}[1]{\oldparagraph{#1}\mbox{}}
\fi
\ifx\subparagraph\undefined\else
\let\oldsubparagraph\subparagraph
\renewcommand{\subparagraph}[1]{\oldsubparagraph{#1}\mbox{}}
\fi

%%% Use protect on footnotes to avoid problems with footnotes in titles
\let\rmarkdownfootnote\footnote%
\def\footnote{\protect\rmarkdownfootnote}

%%% Change title format to be more compact
\usepackage{titling}

% Create subtitle command for use in maketitle
\newcommand{\subtitle}[1]{
  \posttitle{
    \begin{center}\large#1\end{center}
    }
}

\setlength{\droptitle}{-2em}

  \title{Data Structures and Algorithms Homework 1}
    \pretitle{\vspace{\droptitle}\centering\huge}
  \posttitle{\par}
  \subtitle{Due Wednesday Sept 4; Joseph Sepich (jps6444)}
  \author{}
    \preauthor{}\postauthor{}
    \date{}
    \predate{}\postdate{}
  

\begin{document}
\maketitle

\section{Problem 1}\label{problem-1}

I understand the course policies.

\pagebreak

\section{Problem 2}\label{problem-2}

\subsection{Part a. Prove that if a and b are even, then gcd(a, b) =
2gcd(a/2,
b/2)}\label{part-a.-prove-that-if-a-and-b-are-even-then-gcda-b-2gcda2-b2}

\begin{enumerate}
\def\labelenumi{\arabic{enumi}.}
\tightlist
\item
  By the definition of even we can state that a = 2n and b = 2m, where n
  and m are both integers.
\item
  Using the definition in step 1 we can write gcd(a, b) = gcd(2n, 2m).
\item
  Since 2 is a common divisor we can also write gcd(2n, 2m) = 2gcd(n,
  m).
\item
  Using step 1, we also know that n = \(\frac{a}{2}\) and m =
  \(\frac{b}{2}\).
\item
  Plugging step 4 into the equalities in step 2 and 3 we get gcd(a, b) =
  gcd(2n, 2m) = 2gcd(n, m) = 2gcd(\(\frac{a}{2}\), \(\frac{b}{2}\)).
\end{enumerate}

Therefore if a and b are both even, gcd(a, b) = 2gcd(\(\frac{a}{2}\),
\(\frac{b}{2}\)).

\pagebreak

\subsection{Part b. Prove that if a is even and b is odd, then gcd(a, b)
= gcd(a/2,
b)}\label{part-b.-prove-that-if-a-is-even-and-b-is-odd-then-gcda-b-gcda2-b}

\begin{enumerate}
\def\labelenumi{\arabic{enumi}.}
\tightlist
\item
  By the definition of even we can state that a = 2n, where n is an
  integer.
\item
  Using the definition in step 1 we can write gcd(a, b) = gcd(2n, b).
\item
  Since b is odd, it cannot be divided by 2, so the 2 in the term 2n is
  unnecessary information (cannot contribute to the gcd). We can then
  write gcd(a, b) = gcd(n, b).
\item
  Using step 1, we also know that n = \(\frac{a}{2}\).
\item
  Plugging step 4 into the equalities in step 2 and 3 we get gcd(a, b) =
  gcd(2n, b) = gcd(n, b) = gcd(\(\frac{a}{2}\), b).
\end{enumerate}

Therefor if a is even abd b is odd, then gcd(a, b) =
gcd(\(\frac{a}{2}\), b).

\pagebreak

\subsection{Part c. Prove that if a and b are both odd and a
\textgreater{}= b, then gcd(a, b) = gcd((a-b)/2,
b)}\label{part-c.-prove-that-if-a-and-b-are-both-odd-and-a-b-then-gcda-b-gcda-b2-b}

\begin{enumerate}
\def\labelenumi{\arabic{enumi}.}
\tightlist
\item
  By the definition of odd we can state that a = 2n + 1 and b = 2m + 1,
  where n and m are both integers.
\item
  a - b = (2n + 1) - (2m + 1) = 2n - 2m + 1 - 1 = 2(n-m). This must be
  greater than 0, since a \textgreater{}= b.
\item
  a - b is therefore by definition an even number since 2(n - m) can be
  written as 2x = (a-b) where x is the integer n - m.
\item
  Definition of gcd means that d \textbar{} a and d \textbar{} b, where
  d is an integer. This means \(a = d * z\) and \$b = d * y \$ where z
  and y are integers.
\item
  Through step 4 b - a = d(z - y), so d must also divide the integer (z
  - y). This means gcd(a, b) = gcd(a-b, b).
\item
  Since we determined in step 3 a-b is even (and b is even) and in step
  5 gcd(a, b) = gcd(a-b, b), then through the proof in part b of the
  problem we can conclude that gcd(a, b) = gcd(\(\frac{a-b}2\), b).
\end{enumerate}

\pagebreak

\subsection{Part d.}\label{part-d.}

We know that by the defintion of gcd

\begin{verbatim}
int gcd(int a, int b) {
    // input a >= b
    int d = 1;
    if (a == b) return a;
    bool a = isEven(a);  // test parity of variables
    bool b = isEven(b); // unit time
    while (a > 0 && b > 0) {
      if (a is odd and b is even) { // 2 is not a common divisor (part b)
        a = a / 2;
      } else if (a is even and b is odd) { // 2 is not a common divisor (part b)
        b = b / 2;
      } else if (a is odd and b is odd) { // part c
        a = (a - b) / 2; // our input requires a > b
      } else { // both a and b are even, so both can be divded by 2 (as in part a)
        a = a / 2;
        b = b / 2;
        d += 1;
      }
    }
    // d is how many times divided by 2
    // a is non even part of gcd
    return a * 2^d;
}
\end{verbatim}

Now let us asses running time. We are assuming testing parity and
halving are in unit time, so let's focus on subtraction. As we can
recall a positive integer a has at most log(a) bits, and here a is the
larger number. Subtraction two n bit integer taskes O(n) time, so here
it would take us O(log(a)) time. Therefore the algorithm meets the
requirements of the problem.

\pagebreak

\section{Problem 3}\label{problem-3}

\begin{enumerate}
\def\labelenumi{\alph{enumi})}
\item
  \(f(n) = \Omega(g(n))\), this is true because we know with
  polynomials, the higher degree will always grow faster.
\item
  \(f(n) = \Theta(g(n))\), this is true because \(2^{n-1} = 2 * 2^n\)
  and we know that coefficients do not make a difference in larger
  values of n.
\item
  \(f(n) = \Omega(g(n))\), this is true because f(n) has an exponent
  which grows, while g(n) has a constant exponent, therefore f(n) must
  grow faster.
\item
  \(f(n) = \Omega(g(n))\), this is true because in g(n) the
  2\textsuperscript{n} term is dominant. While 2\textsuperscript{n} and
  3\textsuperscript{n} both have the same exponent term of n, the
  integer that has the exponent is greater in f(n).
\item
  \(f(n) = \Theta(g(n))\), this is true because you can transform the
  exponent on each into a coefficient, due to properties of logarithms,
  then change the c you put in front of g(n) to obtain an identical
  function.
\item
  \(f(n) = \Omega(g(n))\), this is true because f(n) is growing a
  constant rate, while g(n) is growing at less than a constant rate.
\item
  \(f(n) = O(g(n))\), this is true because for each additional n, f(n)
  is multiplied by 2, but g(n) is multiplied by n, so it must be growing
  faster.
\item
  \(f(n) = O(g(n))\), this is true because by the logarithm properties
  you are comparing \(nlog(e)\) and \(nlog(n)\), and since we don't look
  at coefficients for growth rates you are comparing n and \(nlog(n)\),
  so while f(n) grows at a constant rate, g(n) is a function that
  increases at an increasing rate.
\item
  \(f(n) = \Theta(g(n))\), this is true because n in each equation is
  the dominating term. This makes both equations \(\Theta(n)\), since
  log(n) grows more slowly than n.
\item
  \(f(n) = \Theta(g(n))\), this is true because of the same concept in
  the last part. n grows faster than both log(n) and
  n\textsuperscript{0.5}. If you chose c to be 5, then they would be
  identical equations.
\end{enumerate}


\end{document}
