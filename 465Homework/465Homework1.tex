\documentclass[]{article}
\usepackage{lmodern}
\usepackage{amssymb,amsmath}
\usepackage{ifxetex,ifluatex}
\usepackage{fixltx2e} % provides \textsubscript
\ifnum 0\ifxetex 1\fi\ifluatex 1\fi=0 % if pdftex
  \usepackage[T1]{fontenc}
  \usepackage[utf8]{inputenc}
\else % if luatex or xelatex
  \ifxetex
    \usepackage{mathspec}
  \else
    \usepackage{fontspec}
  \fi
  \defaultfontfeatures{Ligatures=TeX,Scale=MatchLowercase}
\fi
% use upquote if available, for straight quotes in verbatim environments
\IfFileExists{upquote.sty}{\usepackage{upquote}}{}
% use microtype if available
\IfFileExists{microtype.sty}{%
\usepackage{microtype}
\UseMicrotypeSet[protrusion]{basicmath} % disable protrusion for tt fonts
}{}
\usepackage[margin=1in]{geometry}
\usepackage{hyperref}
\hypersetup{unicode=true,
            pdftitle={Data Structures and Algorithms Homework 1},
            pdfborder={0 0 0},
            breaklinks=true}
\urlstyle{same}  % don't use monospace font for urls
\usepackage{graphicx,grffile}
\makeatletter
\def\maxwidth{\ifdim\Gin@nat@width>\linewidth\linewidth\else\Gin@nat@width\fi}
\def\maxheight{\ifdim\Gin@nat@height>\textheight\textheight\else\Gin@nat@height\fi}
\makeatother
% Scale images if necessary, so that they will not overflow the page
% margins by default, and it is still possible to overwrite the defaults
% using explicit options in \includegraphics[width, height, ...]{}
\setkeys{Gin}{width=\maxwidth,height=\maxheight,keepaspectratio}
\IfFileExists{parskip.sty}{%
\usepackage{parskip}
}{% else
\setlength{\parindent}{0pt}
\setlength{\parskip}{6pt plus 2pt minus 1pt}
}
\setlength{\emergencystretch}{3em}  % prevent overfull lines
\providecommand{\tightlist}{%
  \setlength{\itemsep}{0pt}\setlength{\parskip}{0pt}}
\setcounter{secnumdepth}{5}
% Redefines (sub)paragraphs to behave more like sections
\ifx\paragraph\undefined\else
\let\oldparagraph\paragraph
\renewcommand{\paragraph}[1]{\oldparagraph{#1}\mbox{}}
\fi
\ifx\subparagraph\undefined\else
\let\oldsubparagraph\subparagraph
\renewcommand{\subparagraph}[1]{\oldsubparagraph{#1}\mbox{}}
\fi

%%% Use protect on footnotes to avoid problems with footnotes in titles
\let\rmarkdownfootnote\footnote%
\def\footnote{\protect\rmarkdownfootnote}

%%% Change title format to be more compact
\usepackage{titling}

% Create subtitle command for use in maketitle
\newcommand{\subtitle}[1]{
  \posttitle{
    \begin{center}\large#1\end{center}
    }
}

\setlength{\droptitle}{-2em}

  \title{Data Structures and Algorithms Homework 1}
    \pretitle{\vspace{\droptitle}\centering\huge}
  \posttitle{\par}
  \subtitle{Due Wednesday Sept 4; Joseph Sepich (jps6444)}
  \author{}
    \preauthor{}\postauthor{}
    \date{}
    \predate{}\postdate{}
  

\begin{document}
\maketitle

\section{Problem 1}\label{problem-1}

I understand the course policies.

\pagebreak

\section{Problem 2}\label{problem-2}

\subsection{Part a. Prove that if a and b are even, then gcd(a, b) =
2gcd(a/2,
b/2)}\label{part-a.-prove-that-if-a-and-b-are-even-then-gcda-b-2gcda2-b2}

\begin{enumerate}
\def\labelenumi{\arabic{enumi}.}
\tightlist
\item
  By the definition of even we can state that a = 2n and b = 2m, where n
  and m are both integers.
\item
  Using the definition in step 1 we can write gcd(a, b) = gcd(2n, 2m).
\item
  Since 2 is a common divisor we can also write gcd(2n, 2m) = 2gcd(n,
  m).
\item
  Using step 1, we also know that n = \(\frac{a}{2}\) and m =
  \(\frac{b}{2}\).
\item
  Plugging step 4 into the equalities in step 2 and 3 we get gcd(a, b) =
  gcd(2n, 2m) = 2gcd(n, m) = 2gcd(\(\frac{a}{2}\), \(\frac{b}{2}\)).
\end{enumerate}

Therefore if a and b are both even, gcd(a, b) = 2gcd(\(\frac{a}{2}\),
\(\frac{b}{2}\)).

\pagebreak

\subsection{Part b. Prove that if a is even and b is odd, then gcd(a, b)
= gcd(a/2,
b)}\label{part-b.-prove-that-if-a-is-even-and-b-is-odd-then-gcda-b-gcda2-b}

\begin{enumerate}
\def\labelenumi{\arabic{enumi}.}
\tightlist
\item
  By the definition of even we can state that a = 2n, where n is an
  integer.
\item
  Using the definition in step 1 we can write gcd(a, b) = gcd(2n, b).
\item
  Since b is odd, it cannot be divided by 2, so the 2 in the term 2n is
  unnecessary information (cannot contribute to the gcd). We can then
  write gcd(a, b) = gcd(n, b).
\item
  Using step 1, we also know that n = \(\frac{a}{2}\).
\item
  Plugging step 4 into the equalities in step 2 and 3 we get gcd(a, b) =
  gcd(2n, b) = gcd(n, b) = gcd(\(\frac{a}{2}\), b).
\end{enumerate}

Therefor if a is even abd b is odd, then gcd(a, b) =
gcd(\(\frac{a}{2}\), b).

\pagebreak

\subsection{Part c. Prove that if a and b are both odd and a
\textgreater{}= b, then gcd(a, b) = gcd((a-b)/2,
b)}\label{part-c.-prove-that-if-a-and-b-are-both-odd-and-a-b-then-gcda-b-gcda-b2-b}

\begin{enumerate}
\def\labelenumi{\arabic{enumi}.}
\tightlist
\item
  By the definition of odd we can state that a = 2n + 1 and b = 2m + 1,
  where n and m are both integers.
\item
  a - b = (2n + 1) - (2m + 1) = 2n - 2m + 1 - 1 = 2(n-m). This must be
  greater than 0, since a \textgreater{}= b.
\item
  a - b is therefore by definition an even number since 2(n - m) can be
  written as 2x = (a-b) where x is the integer n - m.
\item
  Definition of gcd means that d \textbar{} a and d \textbar{} b, where
  d is an integer. This means \(a = d * z\) and \$b = d * y \$ where z
  and y are integers.
\item
  Through step 4 b - a = d(z - y), so d must also divide the integer (z
  - y). This means gcd(a, b) = gcd(a-b, b).
\item
  Since we determined in step 3 a-b is even (and b is even) and in step
  5 gcd(a, b) = gcd(a-b, b), then through the proof in part b of the
  problem we can conclude that gcd(a, b) = gcd(\(\frac{a-b}2\), b).
\end{enumerate}

\pagebreak

\subsection{Part d.}\label{part-d.}

We know that by the defintion of gcd

\begin{verbatim}
int gcd(int a, int b) {
    // input a >= b
    int d = 1;
    if (a == b) return a;
    bool a = isEven(a);  // test parity of variables
    bool b = isEven(b); // unit time
    while (a > 0 && b > 0) {
      if (a is odd and b is even) { // 2 is not a common divisor (part b)
        a = a / 2;
      } else if (a is even and b is odd) { // 2 is not a common divisor (part b)
        b = b / 2;
      } else if (a is odd and b is odd) { // part c
        a = (a - b) / 2; // our input requires a > b
      } else { // both a and b are even, so both can be divded by 2 (as in part a)
        a = a / 2;
        b = b / 2;
        d += 1;
      }
    }
    // d is how many times divided by 2
    // a is non even part of gcd
    return a * 2^d;
}
\end{verbatim}

Now let us asses running time. We are assuming testing parity and
halving are in unit time, so let's focus on subtraction. As we can
recall a positive integer a has at most log(a) bits, and here a is the
larger number. Subtraction two n bit integer taskes O(n) time, so here
it would take us O(log(a)) time. Therefore the algorithm meets the
requirements of the problem.

\pagebreak

\section{Problem 3}\label{problem-3}

\begin{enumerate}
\def\labelenumi{\alph{enumi})}
\item
  \(f(n) = \Omega(g(n))\), this is true because we know with
  polynomials, the higher degree will always grow faster.
\item
  \(f(n) = \Theta(g(n))\), this is true because \(2^{n-1} = 2 * 2^n\)
  and we know that coefficients do not make a difference in larger
  values of n.
\item
  \(f(n) = \Omega(g(n))\), this is true because f(n) has an exponent
  which grows, while g(n) has a constant exponent, therefore f(n) must
  grow faster.
\item
  \(f(n) = \Omega(g(n))\), this is true because in g(n) the
  2\textsuperscript{n} term is dominant. While 2\textsuperscript{n} and
  3\textsuperscript{n} both have the same exponent term of n, the
  integer that has the exponent is greater in f(n).
\item
  \(f(n) = \Theta(g(n))\), this is true because you can transform the
  exponent on each into a coefficient, due to properties of logarithms,
  then change the c you put in front of g(n) to obtain an identical
  function.
\item
  \(f(n) = \Omega(g(n))\), this is true because f(n) is growing a
  constant rate, while g(n) is growing at less than a constant rate.
\item
  \(f(n) = O(g(n))\), this is true because for each additional n, f(n)
  is multiplied by 2, but g(n) is multiplied by n, so it must be growing
  faster.
\item
  \(f(n) = O(g(n))\), this is true because by the logarithm properties
  you are comparing \(nlog(e)\) and \(nlog(n)\), and since we don't look
  at coefficients for growth rates you are comparing n and \(nlog(n)\),
  so while f(n) grows at a constant rate, g(n) is a function that
  increases at an increasing rate.
\item
  \(f(n) = \Theta(g(n))\), this is true because n in each equation is
  the dominating term. This makes both equations \(\Theta(n)\), since
  log(n) grows more slowly than n.
\item
  \(f(n) = \Theta(g(n))\), this is true because of the same concept in
  the last part. n grows faster than both log(n) and
  n\textsuperscript{0.5}. If you chose c to be 5, then they would be
  identical equations.
\end{enumerate}

\pagebreak

\section{Problem 4}\label{problem-4}

\subsection{Part a.}\label{part-a.}

\begin{enumerate}
\def\labelenumi{\arabic{enumi}.}
\tightlist
\item
  Recall we talked about multiplication in lecture. Using our standard
  multiplication algorithm it takes \(\Theta(n^2)\) time. This is
  because for each new digit, another row is added from having to
  multiply another digit with the second number. You also then have to
  add up the new parts in the addition part of the algorithm.
\item
  Since we are summing up 1 through n, we are doing the multiplcation
  (squaring step) from step 1 n times. This means we are going to be
  performing n*n\textsuperscript{2} operations.
\end{enumerate}

Therefore the running time must be \(\Theta(n^3)\).

\pagebreak

\subsection{Part b.}\label{part-b.}

\begin{enumerate}
\def\labelenumi{\arabic{enumi}.}
\tightlist
\item
  For this statement we can expand on the previous proof.
\item
  Agains we are doing n operations of k\textsuperscript{j}.
\item
  k\textsuperscript{j} must be \(\Theta(n^j)\), because performing a
  multiplication of x and x (to be y) will be n\textsuperscript{2}.
\item
  Once you have the answer to x andx from step 3, you can say that if
  you have k\textsuperscript{j} where j is a multiple of two, you can
  dececrease it to (x and x)\textsuperscript{j/2}, and so on. You can do
  this until you have to multiply by x again, if the exponent is odd.
  This would make it so you have n\textsuperscript{2} running time,
  making it j operations.
\item
  If we combine step 4 and 2 we can conclude that the summation will
  take \(\Theta(n^{j+1})\) running time.
\end{enumerate}

\pagebreak

\subsection{Part c.}\label{part-c.}

\begin{enumerate}
\def\labelenumi{\arabic{enumi}.}
\tightlist
\item
  The summation in this problem is also known as the harmonic series.
\item
  We know that the integral of \(\frac1{x} = ln(n)\), so we can use this
  to show that the partial sum of the series is bound by ln(n).
\item
  The sum of
  \(\frac{1}{x+1} \leq \int\frac1{x}dx = ln(n) \leq \Sigma \frac1{x}\).
\end{enumerate}

Therefore the harmonic series must be bound as \(\Theta(log(n))\).

\pagebreak

\subsection{Part d.}\label{part-d.-1}

\begin{enumerate}
\def\labelenumi{\arabic{enumi}.}
\tightlist
\item
  By definition \(log(n!) = log(1) + log(2) + ... + log(n)\).
\item
  Upper bound we know that \(log(n!)\) as written above
  \(\leq log(n) + log(n) + log(n) = n*log(n)\).
\item
  The partial sum of \(log(n!) = log(n/2) + ... + log(n)\) is then
  \(\leq log(n!)\), which we know by definition of partial sum.
\item
  We can also say the given partial sum
  \(\geq log(n/2) + log(n/2) + ... + log(n/2) = n/2 * log(n/2)\), which
  follows the \(n * log(n)\) structure.
\end{enumerate}

Therefore since \(log(n!)\) is upper and lower bounded by \(nlog(n)\) it
must be \(\Theta(n*log(n))\).

StackOverflow was referenced to completed this problem.

\pagebreak

\subsection{Part e.}\label{part-e.}

\begin{enumerate}
\def\labelenumi{\arabic{enumi}.}
\tightlist
\item
  In the summation, c is a constant, where i (its exponent) is being
  incremented in the summation.
\item
  If c \textgreater{} 1, the series is increasing and follows the proof
  in part b of this problem. By the statement in part b the running time
  must be \(\Theta(c^k)\), with no + 1, since the series starts at 0.
\item
  If c == 1, the running time is linear, since one is merely summing up
  1, k times.
\item
  If c \textless{} 1, then c is will create a similar series to the
  harmonic series, but the series will be decreasing and convergent, so
  one knows the summation equation to sum the series, making it unit
  time.
\end{enumerate}

\pagebreak

\section{Problem 5}\label{problem-5}

\subsection{1.}\label{section}

\begin{itemize}
\tightlist
\item
  An outer loop does n iterations
\item
  An inner loop does (n-i) / 5 iterations
\end{itemize}

The second bullet point is true, since j is initialized to be i, and is
incremented by 5 each time. These two statements imply a running time of
n * (n-i) / 5 = \(\frac15n^2-ni\). We know that in this running time the
n\textsuperscript{2} term dominates, so it is \(\Theta(n^2)\).

\pagebreak

\subsection{2.}\label{section-1}

\begin{itemize}
\tightlist
\item
  An outer loop does n iterations
\item
  An inner loop does (n-4i) iterations
\end{itemize}

The second bullet point is true, since j is intialized to 4i, but still
increments only 1 every loop. Since it is intialized to 4i, it will not
run those 4i times. These two statements imply a running time of
\(n * (n-4i) = n^2-4ni\), which is a running time of \(\Theta(n^2)\),
with the dominant n\textsuperscript{2} term.

\pagebreak

\subsection{3.}\label{section-2}

\begin{itemize}
\tightlist
\item
  An outer loop does n iterations
\item
  An inner loop does n-i\textsuperscript{5} iterations
\end{itemize}

You could flip the algorithm to have the same running time where the
inner loops says:

\begin{verbatim}
j := i^5;
while j < n do
  j := j + 1;
end
\end{verbatim}

This means we once again have \(\Theta(n^2)\), since the running time is
\(n*(n-i^5)=n^2-ni^5\).

\pagebreak

\subsection{4.}\label{section-3}

\begin{itemize}
\tightlist
\item
  An outer loop does n iterations
\item
  An inner loop does \(\frac{log_2(n)}{4}\) iterations.
\end{itemize}

The second bullet point is true, because j starts at 2, increments by
itself to the 4th, but only goes until i. This means it will go from 2
to 16 to 256 to 4096 and so on. These two statements imply
\(n * \frac{log_2(n)}{4} = \Theta(nlog(n))\)


\end{document}
